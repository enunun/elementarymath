\documentclass{ltjsarticle}
%
\usepackage{amsmath} % 数式用
\usepackage{mleftright} % カッコ
\usepackage{cleveref} % 相互参照
%
\newcommand{\paren}[1]{\mleft ( #1 \mright )} % 丸カッコ
\newcommand{\apply}[2]{#1 \paren{#2}} % 関数適用
\newcommand{\derivprime}[1]{#1 ^{\prime}}
\newcommand{\derivprimetwo}[1]{#1 ^{\prime \prime}}
%
\crefname{equation}{式}{式}
%
\begin{document}
%
\section{接線の方程式}
\label{sec:tangentline}

関数$\apply{f}{x}$が点$a$で微分可能であるとき,
その点における曲線$y = \apply{f}{x}$接線の方程式は\cref{eq:tangentline}のようになる.
\begin{equation}
  y = \apply{f}{a} + \apply{\derivprime{f}}{a} \paren{x - a} .
  \label{eq:tangentline}
\end{equation}
$x$が十分$a$に近いとき,曲線$y = \apply{f}{x}$とこの接線がほぼ一致するとみなせるので,
近似式
\begin{equation}
  \apply{f}{x} \approx \apply{f}{a} + \apply{\derivprime{f}}{a} \paren{x - a}
  \label{eq:approxequation1deg}
\end{equation}
を得る.

\cref{eq:approxequation1deg}は与えられた関数を1次式で近似するための式であるが,
2次式で近似したければ以下のようにすればよい.

\begin{equation}
  \apply{f}{x} \approx \apply{f}{a} + \apply{\derivprime{f}}{a} \paren{x - a} 
  + \frac{\apply{\derivprimetwo{f}}{a}}{2} \paren{x - a}^2.
  \label{eq:approxequation2deg}
\end{equation}

\end{document}
